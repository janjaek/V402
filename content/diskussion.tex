\section{Diskussion}
\label{sec:Diskussion}
Wie an der Unsicherheit von $\phi$, welche unter \SI{0.04}{\percent} liegt, zu sehen ist, kann mittels eines Goniometers ein Winkel sehr genau bestimmt werden.
Somit können auch die Brechungsindices sehr genau bestimmt werden und der Unsicherheit liegt im Berech von \SI{0.01}{\percent}.
Anhand der Graphen der optimierten Näherungsfunktionen, lässt sich bereits klar sagen, dass Gleichung \eqref{eq:f2} nicht geeignet ist, um die Dispersion zu nähern.
Die Fehlerquadrate unterstützen dies deutlich, durch einen Unterschied von vier Größenordnungen.
Die Absorbtionsfrequenz lässt sich aus den Parametern der Dispersionsfunktion bestimmen. 
Da dies auf zwei Wege möglich ist und beide Ergebnisse nah beienander liegen, ist davon auszugehen, dass sich der tatsächliche Wert nah bei diesen befindet.
Jedoch ist anzumerken, dass unter Verwendung der Koeffizienten höherer Ordnung eine größere Unsicherheit auftritt.
Die Abbesche Zahl lässt sich leider nicht direkt aus den Messwerten bestimmen, da aufgrund der Art der Lichtquelle die Fraunhoferlinien nicht zu beobachten waren.
Jedoch lässt sie sich mit relativ geringer Unsicherheit aus der Dispersionsfunktion bestimmen.
Auch das Auflösungsvermögen kann mit kleiner Unsicherheit bestimmt werden.
